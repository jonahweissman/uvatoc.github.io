\documentclass[11pt]{article}
\usepackage{uvatoc} % replace this line with the one below for your submission
%\usepackage[response]{uvatoc}

\begin{document}

\makeheader

\makemytitle{Problem Set 0: Getting Started}
\submitter{TODO: your name}
\due{4:19pm, Friday, 6 September}

\directions{
This assignment is a ``warm-up'' to get familiar with the tools and submission processes we will use in cs3102, and to refresh your memory of some things you should already be familiar with from the course prerequisites.

\collaboration{You should do this assignment by yourself and submit your own answers. You may discuss the problems with anyone you want and it is also fine to get help from anyone on problems with LaTeX or Jupyter/Python. You are permitted to use any resources you find for this assignment. You should note in the {\em Collaborators and Resources} box below the people you collaborated with and any external resources you used.}
}

\collaborators{TODO: replace this with your collaborators and resources (if you did not have any, replace this with {\em None})}

\directions{
This problem set has two components designed to get you familiar with the tools we will use for assignments in cs3102: Jupyter notebooks and LaTex.

\section{Getting Started with LaTeX}

For the assignments in this class, you will be required to submit your responses as PDF files typeset with LaTeX\footnote{To Quote Leslie Lamport (the creator LaTeX) ``One of the hardest things about LaTeX is deciding how to pronounce it. This is also one of the few things I'm not going to tell you about LaTeX, since pronunciation is best determined by usage, not fiat. TeX is usually pronounced teck, making lah-teck, and lay-teck the logical choices; but language is not always logical, so lay-tecks is also possible.''}, a professional formatting system that is used in most serious mathematical typesetting, which is a set of libraries built on the TeX typesetting language developed by Donald Knuth.

If you haven't used LaTeX before, there is a bit of a learning curve to using it, but you will find the ability it gives you to efficiently produce beautiful and complex documents to be a valuable life-long skill. We recommend using \hyperlink{www.overleaf.com}{Overleaf}, an in-browser collaborative editor for LaTeX.

\subsection{Register for Overleaf}

Visit \url{https://www.overleaf.com} and register for an Overleaf account (if you don't already have one). UVA has a site license to Overleaf, so if you register with your \keyword{@virginia.edu} email address you will have full access to all the Overleaf features for free.

\subsection{Clone the PS0 Repository}

Create your own copy of the Problem Set 0 repository, by following these steps:

\begin{enumerate}
\item Download the Problem Set 0 template from: \url{https://uvatoc.github.io/ps/ps0.zip}
\item In Overleaf, click on \keyword{Create First Project} or \keyword{New Project} in Overleaf and select \keyword{Upload Project} from the menu.
\item Click \keyword{Select a .zip file} and then select the \keyword{ps0.zip} file you downloaded in step 1.
\end{enumerate}

In the left side of the browser, you should see a file directory containing \keyword{ps0.ipynb}, a Jupyter notebook that you'll use in the second part (but shouldn't edit or view in Overleaf), \keyword{ps0.tex}, the template you will modify for this problem set, and \keyword{uvatoc.sty}, a style file that defines useful macros for cs3102 (you are welcome to look at this file, but should not need to modify it). You can click on \keyword{ps0.tex} to see the LaTeX source for this file.

Click \keyword{Recompile} to build the PDF. You should see this document in the right side of the browser.

\subsection{Editing ps0.tex}

The first thing you should do set up your name as the author of the submission:
\begin{itemize}
    \item Look for the line, \texttt{\textbackslash submitter\{TODO: your name\}} and replace the {\em TODO: your name}\ with your name and UVA id:
    \begin{quote}
        \texttt{\textbackslash submitter\{Grace Hopper (gmh1a)\}}
    \end{quote}
\item List your collaborators and resources, replacing the TODO in {\texttt{\textbackslash collaborators\{TODO: replace ...\}}} with your collaborators and resources. (Remember to update this before submitting if you work with more people.)

\item Replace the line, \texttt{\textbackslash usepackage\{uvatoc\}} (the second line in the file) with \texttt{\textbackslash usepackage[response]\{uvatoc\}}. You can do this by using the LaTeX comment token, {\texttt{\%}}. The rest of the line after a {\texttt{\%}} is treated as a comment. 
\end{itemize}
Then, try rebuilding the PDF by clicking \keyword{Recompile}. You should see a file that includes your name and collaborators, but with all the directions removed (we don't want to see these again in your submission).
}

\begin{problem}
I cannot live without...\end{problem}
\directions{
Include your favorite passage from a book. Cite the source as a resource above.}

% Your answer for problem here

\begin{problem}The Finest Gambit\end{problem}
\directions{
\begin{quote}
\textit{Reductio ad absurdum}, which Euclid loved so much, is one of a mathematician's finest weapons. It is a far finer gambit than any chess play: a chess player may offer the sacrifice of a pawn or even a piece, but a mathematician offers the game. [Excerpt from {\em A Mathematician's Apology}, G.H. Hardy, 1940, p. 94]  
\end{quote}

\noindent Learn how to write math and construct proofs by reproducing the proof below. You will need to use the ``align'' environment, as well as the ``align*'' environment.

\begin{definition}
A rational number is a fraction $\frac{a}{b}$ where $a$ and $b$ are integers. 
\end{definition}

\noindent Show $\sqrt{2}$ is irrational.

\begin{proof}

TODO: Put your proof here

\end{proof}

}

% Your answer for problem here

\begin{problem} Vanity \end{problem}
\directions{
Learn how to include drawings in your documents with the {\tt \textbackslash includegraphics\{file\}} command by submitting a caricature of David Evans and/or Nathan Brunelle.
}

% Your answer for problem here

\directions{
\section{Getting started with Jupyter}

For this part of the assignment, you will get started using Jupyter notebooks. 

The \keyword{ps0.zip} file includes \keyword{ps0.ipynb}, a Jupyter notebook which combines text (in Markdown format) and executable Python code.

To execute and edit the Jupyter notebook, you will need to install jupyter. Visit \url{https://jupyter.org/} and follow the directions at \url{https://jupyter.org/install.html} to install jupyter on your machine. (As in the directions there, we recommend using Anaconda, which will also install Python and many useful Python packages. The instructions in the project notebook assume you have installed Anaconda.)

Once you’ve installed Jupyter, run
\begin{quote}
\texttt{jupyter notebook ps0.ipynb}
\end{quote}
to get started. This will start the jupyter local server and open the notebook in your web browser. 

The notebook contains the Problem 4 and 5 and starting code for this assignment. Follow the directions there, and complete . You will do your assignment by editing this file, and will submit your completed jupyter notebook as your assignment.
}

\setcounter{problem}{5} % This updates the problem number

\begin{problem} Python vs.\ Math\end{problem}

The implementation of the {\tt mset} datatype in the Jupyter notebook is meant to represent a mathematical set, but is different from a mathematical set in several important ways. Describe at least two ways the Python {\texttt mset} datatype differs from the mathematical set definition.


\end{document}
